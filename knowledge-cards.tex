\documentclass{knowledge-cards}
  
%----------------------------------------------------------------------
\begin{document}
%----------------------------------------------------------------------
  
%----------------------------------------------------------------------
\card%
{specification}%
{O10}%
{The knowledge-card \LaTeX\ class}%
{20180729}%
{https://github.com/mariomerinomartinez/knowledge-cards}
%----------------------------------------------------------------------

This is the documentation of the
\cc{knowledge-card} \LaTeX\ class\sidenote{\link{https://github.com/ mariomerinomartinez/ knowledge-cards}{https://github.com/mariomerinomartinez/knowledge-cards}}, 
which is a custom subclass of thet \cc{tufte-latex} class\sidenote{\link{https://ctan.org/pkg/tufte-latex}{https://ctan.org/pkg/tufte-latex}}. The code of this card can be used as a template when writing new cards.

%----------------------------------------------------------------------
\section{Card definition}
%----------------------------------------------------------------------
 
Define a new card type in the \bb{preamble} 
with\marginnote{\cc{\textbackslash newcardtype}}
%
{ \color{\cardtype}%
\begin{verbatim}
  \newcardtype{card type}{R, G, B}
\end{verbatim}
}%
%
\noindent where \cc{R,G,B} range from $0$ to $255$ and 
define the color of the card.
Then, after {\color{\cardtype} \verb|\begin{document}|}, 
declare a new card with\marginnote{\cc{\textbackslash card}}
%
{\color{\cardtype}%
\begin{verbatim}
  \card{card type}{card label}{card title}{date}{card URL}
    <your text goes here> 
  \newpage ~ % Put this only if card is one page
\end{verbatim}
}%
%
\noindent 
The \cc{card URL} is used to format the QR code in the back page
of the card.
The variable \cc{\textbackslash cardtype}\marginnote{\cc{\textbackslash cardtype}} is set automatically after the card declaration and can be used
to refer to the card color. 
Cards only accept {\color{\cardtype}\verb|\section|}
and {\color{\cardtype}\verb|\subsection|} for sectioning.

%----------------------------------------------------------------------
\section{Text styles}
%----------------------------------------------------------------------
 
%
\begin{itemize}
\item {\color{\cardtype}\verb|\bb|}: bold colorized font\marginnote{
\cc{\textbackslash bb}, \cc{\textbackslash ii}, \cc{\textbackslash cc}}.
\item {\color{\cardtype}\verb|\ii|}: italic colorized font.
\item {\color{\cardtype}\verb|\cc|}: typewriter colorized font.
\item {\color{\cardtype}\verb|\link{link text}{address}|}: colorized link. Use
\cc{mailto:} to create a link to an email address\marginnote{
\cc{\textbackslash link}, \cc{\textbackslash cardlink}}.
\item {\color{\cardtype}\verb|\cardlink{card type}{link text}{address}|}: colorized link to another card, of type \cc{card type}.  
\end{itemize}
%

%----------------------------------------------------------------------
\section{Using the margin}
%----------------------------------------------------------------------
 
Side notes and references: 
%
\begin{itemize}
\item {\color{\cardtype}\verb|\sidenote[<number>][<offset>]{text}|}: numbered side note\marginnote{\cc{\textbackslash sidenote}, \cc{\textbackslash marginnote}}.
\item {\color{\cardtype}\verb|\marginnote[<offset>]{text}|}: unnumbered side note.
\item {\color{\cardtype}\verb|\cite[<offset>]{bibkey1,bibkey2,...}|}: cite\marginnote{\cc{\textbackslash cite}} references in bibtex file.
\end{itemize}
%
Figures\marginnote{\cc{\textbackslash marginfigure}}:
%
\begin{marginfigure}[+0.1cm]
\begin{center}
\includegraphics[width=\textwidth]{figs/latex.png} 
\end{center}
\caption{Example of margin figure.}
\label{fig:examplefig}
\end{marginfigure}
%
%
{\color{\cardtype}%
\begin{verbatim}
  \begin{marginfigure}[<offset>]
    \begin{center}
      \includegraphics{figurefile} 
    \end{center}
    \caption{text}
    \label{figlabel}
  \end{marginfigure}
\end{verbatim}
}%
%
Margin tables can be likewise created with\marginnote{\cc{\textbackslash margintable}} \cc{\textbackslash margintable}.

The optional \cc{offset} argument in the commands and 
environments above allows moving
the margin object up (e.g. \cc{+1cm}) or down (e.g. \cc{-1cm}), 
useful if there are overlaps with other
margin objects. 
The optional \cc{number} argument in {\color{\cardtype}\verb|\sidenote|}
changes the default side note number. 
To give an offset but leave the number unchanged, use
{\color{\cardtype}\verb|\sidenote[][<offset>]{text}|}.
To be able to cite references, load the bibtex database somewhere before
{\color{\cardtype}\verb|\end{document}|}
with {\color{\cardtype} \verb|\nobibliography{bibfile}|}.
With figures and tables, the caption and label are optional.

To create text, a figure, or a table that
\bb{spans the main text area and the margin}, use the
\cc{fullwidth},\marginnote{\cc{\textbackslash fullwidth}, \cc{\textbackslash figure*}, \cc{\textbackslash table*}}
\cc{figure*} and
\cc{table*} environments.
Figures and tables in the main text area are made with the standard
\cc{figure} and \cc{table} environments.
%
\begin{figure}
\begin{center}
\includegraphics[width=\textwidth]{figs/latex.png} 
\end{center}
\caption{Example figure.}
\label{fig:examplefig2}
\end{figure}
%
%
\begin{figure*}
\begin{center}
\includegraphics[width=\textwidth]{figs/latex.png} 
\end{center}
\caption{Example figure*.}
\label{fig:examplefig3}
\end{figure*}
%
 
% \newpage ~

%----------------------------------------------------------------------
\nobibliography{bibtex/bibtex} 
%----------------------------------------------------------------------

%----------------------------------------------------------------------
\end{document}
%----------------------------------------------------------------------